%
% resume.tex
% @author Omar Moreno
% @date January 09, 2016
%

\documentclass[10pt]{article}

%%%%%%%%%%%%%%%%
%   Packages   %
%%%%%%%%%%%%%%%%

\usepackage[usenames,dvipsnames]{xcolor}
\usepackage[top=0.75in, bottom=0.75in, left=0.75in, right=0.75in]{geometry}
\usepackage{enumitem}
\usepackage[none]{hyphenat}
\usepackage[hidelinks]{hyperref}
\usepackage{fontawesome}
\pagenumbering{gobble}

%%%%%%%%%%%%%%%%%%%%%%%
%   Custom Commands   %
%%%%%%%%%%%%%%%%%%%%%%%

% Use the font Computer Modern Sans Serif
\renewcommand*{\familydefault}{\sfdefault}

% Define Dim Gray color 
\definecolor{dimgray}{HTML}{696969}
\definecolor{indigodye}{HTML}{0D4F8B}

%
% resumesection : Command used to denote a new section of the CV
% 
% #1 Name of the section
%
\newcommand{\resumesection}[1] {
    \noindent
    \textcolor{indigodye}{\rule{.15\textwidth}{.1in} \hspace{0.01 \textwidth} \textbf{\Large{#1}}} \newline
}

\newcommand{\experienceentry}[5] { 
    \noindent
    \begin{minipage}[t]{0.15\textwidth} \begin{flushright} #1 \end{flushright} \end{minipage} \hspace{0.01\textwidth}
    \begin{minipage}[t]{0.84\textwidth} 
        \textbf{#2}, \emph{#3}, #4 
        #5
    \end{minipage}
}

\newcommand{\educationentry}[4] { 
    \noindent
    \begin{minipage}[t]{0.15\textwidth} \begin{flushright} #1 \end{flushright} \end{minipage} \hspace{0.01\textwidth}
    \begin{minipage}[t]{0.84\textwidth} 
        \textbf{#2}, \emph{#3}, #4
    \end{minipage}
}

\newcommand{\skillsentry}[2] { 
    \noindent
    \begin{minipage}[t]{0.15\textwidth} \begin{flushright} #1 \end{flushright} \end{minipage} \hspace{0.01\textwidth}
    \begin{minipage}[t]{0.84\textwidth} #2 \end{minipage}
}

\newcommand{\awardentry}[2] { 
    \noindent
    \begin{minipage}[t]{0.15\textwidth} \begin{flushright} #1 \end{flushright} \end{minipage} \hspace{0.01\textwidth}
    \begin{minipage}[t]{0.84\textwidth} #2 \end{minipage}
}

\renewcommand{\section}[2]{}

\begin{document}

    %%%%%%%%%%%%%%
    %   Header   %
    %%%%%%%%%%%%%%
    \noindent
    \begin{minipage}[c]{0.5\textwidth}
        \begin{flushleft}
            \Huge{Omar Moreno}
        \end{flushleft}
    \end{minipage}
    \begin{minipage}[c]{0.50\textwidth}
        \begin{flushright}
            \color{dimgray} \em
            Sunnyvale, CA           \\
            \faMobilePhone \hspace{1pt} +1 (562) 396-1622       \\
            \faEnvelope \hspace{1pt} \href{mailto:email@omarmoreno.net}{email@omarmoreno.net}             \\
            \faLinkedin \hspace{1pt} \href{https://www.linkedin.com/in/omarmoreno2}{omarmoreno2}          \\
            \faGithub \hspace{1pt} \href{https://github.com/omar-moreno}{omar-moreno}                     \\
        \end{flushright}
    \end{minipage}

    \resumesection{Experience}
        \experienceentry{2009-present}
                        {Graduate Student Researcher}
                        {Santa Cruz Institute for Particle Physics}
                        {Santa Cruz, CA}
                        {
                            \begin{itemize}[label=\textcolor{indigodye}{$\circ$}, noitemsep, nolistsep, leftmargin=*]
                              \item Member of the Heavy Photon Search (HPS) Collaboration consisting of over 50 
                                    physicists and engineers.
                              \item Used frequentist statistical analysis to conduct a resonance search for a new
                                    fundamental particle, heavy photon, thought to mediate dark matter 
                                    interactions.
                              \item Developed maximum likelihood fitter to determine parameters of model describing 
                                    Quantum Electrodynamics Trident background.
                              \item Optimized selection of radiative (signal like) events using a Random Forest 
                                    classifier, boosting signal/background fraction by 40\%.
                              \item Co-developer of a Java object pipeline used to process and clean up
                                    over 5 TB of noisy data from the HPS Silicon Vertex Tracker (SVT) into basic
                                    physics objects used for analysis by HPS users.
                              \item Lead developer of a C++ package used to create and persist complex physics 
                                    objects in ROOT data structures.
                              \item Developed several Java applications used to monitor the online performance
                                    of the HPS SVT.
                              \item Developed C++ package used to characterized the performance of several components
                                    of the HPS SVT, extract calibration constants and write them to XML.
                              \item Developed Java front end used to load greater than 100,000 SVT calibration 
                                    constants to a MySQL database.  
                              \item Key contributor to the design, installation, testing and operation of both the 
                                    test and engineering run HPS data acquisition system.
                          \end{itemize}
                        }
        \experienceentry{2007-2009}
                        {Graduate Student Researcher}
                        {Department of Physics and Astronomy, California State University, Los Angeles}
                        {Los Angeles, CA}
                        {
                          \begin{itemize}[label=\textcolor{indigodye}{$\circ$}, noitemsep, nolistsep, leftmargin=*]
                            \item Member of the GEP-III Collaboration consisting of over 50 physicists and engineers.
                            \item Optimized detector selection resulting in an improved measurement of the analyzing
                                powers of the reaction $p + \mbox{CH}_2 \rightarrow X$ at $Q^2 = 2.733$ GeV/c$^{2}$.
                            \item Measured the form factor ratio, $G_{E_p}/G_{M_p}$, of the proton at 
                                  a $Q^{2} = 2.733$ GeV$^{2}$.
                          \end{itemize}
                        } 
        \experienceentry{2005-2006}
                        {Undergraduate Researcher}
                        {Department of Physics and Astronomy, University of California, Irvine}
                        {Irvine, CA}
                        { 
                          \begin{itemize}[label=\textcolor{indigodye}{$\circ$}, noitemsep, nolistsep, leftmargin=*]
                            \item Member of the BaBar Collaboration, consisting of over 600 physicists and engineers.
                            \item Developed C++ analysis to measure the branching fraction for the rare decay 
                                  $B\rightarrow e^+e^-$
                            \item Used a multilayer perceptron to boost the identification of the decay 
                                  $\Lambda \rightarrow p \pi^-$ by 10\%.  
                          \end{itemize}
                       }
        \experienceentry{2000-2001}
                        {Mechanical Engineering Apprentice}
                        {Nasa Dryden Flight Research Center}
                        {Edwards, CA}
                        {
                          \begin{itemize}[label=\textcolor{indigodye}{$\circ$}, noitemsep, nolistsep, leftmargin=*]
                            \item Designed and constructed a device used to evaluate the skin-friction reduction 
                                  of several Micro-Blowing Technique skins at supersonic speeds. 
                          \end{itemize}
                       }

    \resumesection{Education}
        \educationentry{(Expected) 2016}
                       {Ph.D. in Physics}
                       {University of California at Santa Cruz}
                       {Santa Cruz, CA}
        \educationentry{2007-2009}
                       {M.Sc. in Physics}
                       {California State University, Los Angeles}
                       {Los Angeles, CA}
        \educationentry{2001-2006}
                       {B.Sc. in Applied Physics}
                       {University of California at Irvine}
                       {Irvine, CA}

    \resumesection{Skills}
    \skillsentry{Languages}{Java, C++, C, Python, MySQL, XML, Mathematica. Some experience with HTML5 and Fortran}
        \skillsentry{Tools}{ROOT, RooFit, scikit-learn, NumPy, Matplotlib, IPython, git, SVN, Linux, LaTex, CMake}

    \resumesection{Awards}
        \awardentry{2012}{Margaret Burbidge Award for Best Experimental Research}
        \awardentry{2009}{Margaziotis Award for Best Experimental Research}
        \awardentry{2007-2009}{Louis Stokes Alliance for Minority Participation Bridge to the Doctorate 
                               Fellowship}
        \awardentry{2006}{California Alliance for Minority Participation Mentor of the Year}
        \awardentry{2001-2002}{Chancellor's Leadership Scholar}

\end{document}
