%
% resume.tex
% @author Omar Moreno
% @date January 09, 2016
%

\documentclass[11pt]{article}

%%%%%%%%%%%%%%%%
%   Packages   %
%%%%%%%%%%%%%%%%

\usepackage[usenames,dvipsnames]{xcolor}
\usepackage[top=0.75in, bottom=0.75in, left=0.75in, right=0.75in]{geometry}
%\usepackage{paralist}
\usepackage{enumitem}
\usepackage[none]{hyphenat}
\pagenumbering{gobble}

%%%%%%%%%%%%%%%%%%%%%%%
%   Custom Commands   %
%%%%%%%%%%%%%%%%%%%%%%%

% Use the font Computer Modern Sans Serif
\renewcommand*{\familydefault}{\sfdefault}

% Define Dim Gray color 
\definecolor{dimgray}{HTML}{696969}

%
% resumesection : Command used to denote a new section of the CV
% 
% #1 Name of the section
%
\newcommand{\resumesection}[1] {
    \vspace{20pt}
    \noindent
    \textcolor{MidnightBlue!50!White}{\rule{.15\textwidth}{5pt} \hspace{0.05} \textbf{\large{#1}}} \newline
}

\newcommand{\experienceentry}[5] { 
    \noindent
    \begin{minipage}[t]{0.15\textwidth} \begin{flushright} #1 \end{flushright} \end{minipage} \hspace{0.05}
    \begin{minipage}[t]{0.85\textwidth} 
        \textbf{#2}, \emph{#3}, #4 
        #5
    \end{minipage}
}

\newcommand{\educationentry}[4] { 
    \noindent
    \begin{minipage}[t]{0.15\textwidth} \begin{flushright} #1 \end{flushright} \end{minipage} \hspace{0.05}
    \begin{minipage}[t]{0.85\textwidth} 
        \textbf{#2}, \emph{#3}, #4
    \end{minipage}
}

\newcommand{\skillsentry}[2] { 
    \noindent
    \begin{minipage}[t]{0.15\textwidth} \begin{flushright} #1 \end{flushright} \end{minipage} \hspace{0.05}
    \begin{minipage}[t]{0.85\textwidth} #2 \end{minipage}
}

\renewcommand{\section}[2]{}

\begin{document}

    %%%%%%%%%%%%%%
    %   Header   %
    %%%%%%%%%%%%%%
    \noindent
    \begin{minipage}[c]{0.5\textwidth}
        \begin{flushleft}
            \Huge{Omar Moreno}
        \end{flushleft}
    \end{minipage}
    \begin{minipage}[c]{0.50\textwidth}
        \begin{flushright}
            \color{dimgray} \em
            Sunnyvale, CA           \\
            +1 (562) 396-1622       \\
            email@omarmoreno.net    \\
            omarmoreno2             \\
            omar-moreno             \\
        \end{flushright}
    \end{minipage}

    \resumesection{Experience}
        \experienceentry{2009-present}
                        {Graduate Student Researcher}
                        {Santa Cruz Institute for Particle Physics}
                        {Santa Cruz, CA}
                        {
                          \begin{itemize}[noitemsep, nolistsep, leftmargin=*]
                              \item Member of the Heavy Photon Search (HPS) Collaboration
                              \item Using frequentist statistical analysis to conduct a resonance search for a new
                                    fundamental particle, heavy photon, thought to mediate dark matter 
                                    interactions.
                              \item One of three lead developers of a Java object pipeline used to process process
                                    noisy data from the HPS Silicon Vertex Tracker (SVT) into basic physics objects
                                    used by HPS users.
                              \item Lead developer of a C++ packaged used to create and persist complex physics 
                                    objects in ROOT data structures.
                              \item Developed several Java applications used to monitor the real time performance
                                    of the HPS SVT.
                              \item Developed C++ package used to characterized the performance of several components
                                    of the HPS SVT, extract calibration constants and write them to XML.
                              \item Developed Java front end used to load and retrieve SVT calibration constants to a
                                    MySQL database.
                              \item Key contributor to the design, installation, testing and operation of both the 
                                    test and engineering run HPS data acquisition system.
                              \item Characterized the performance of the Long Shaping Time Front End readout chip at
                                    various stages of development.
                          \end{itemize}
                        }
        \experienceentry{2007-2009}
                        {Graduate Student Researcher}
                        {Department of Physics and Astronomy, California State University, Los Angeles}
                        {Los Angeles, CA}
                        {
                          \begin{itemize}[noitemsep, nolistsep, leftmargin=*]
                            \item Improved the analyzing powers of the reaction 
                                  $p + \mbox{CH}_2 \rightarrow X$ at $Q^2 = 2.27$ GeV/c.
                            \item Measured the form factor ratio, $G_{E_p}/G_{M_p}$, of the proton. 
                          \end{itemize}
                        } 
        \experienceentry{2005-2006}
                        {Undergraduate Researcher}
                        {Department of Physics and Astronomy, University of California, Irvine}
                        {Irvine, CA}
                        { 
                          \begin{itemize}[noitemsep, nolistsep, leftmargin=*]
                            \item Improved the branching fraction measurement for the rare decay 
                                  $B\rightarrow e^+e^-$ by a factor of 6.
                            \item Used a multilayer perceptron to improve particle identification in the 
                                  $\Lambda \rightarrow p \bar{p}$ decay channel by 10\%.  
                          \end{itemize}
                       } \newline
        \experienceentry{2000-2001}
                        {Mechanical Engineering Apprentice}
                        {Nasa Dryden Flight Research Center}
                        {Edwards, CA}
                        {
                          \begin{itemize}[noitemsep, nolistsep, leftmargin=*]
                            \item Designed and constructed a device used to evaluate the skin-friction reduction 
                                  of several Micro-Blowing Technique skins at supersonic speeds. 
                          \end{itemize}
                       }

    \resumesection{Education}
        \educationentry{(Expected) 2016}
                       {Ph.D. Candidate in Physics}
                       {University of California at Santa Cruz}
                       {Santa Cruz, CA}
        \educationentry{2007-2009}
                       {M.Sc. in Physics}
                       {California State University, Los Angeles}
                       {Los Angeles, CA}
        \educationentry{2001-2006}
                       {B.Sc. in Applied Physics}
                       {University of California at Irvine}
                       {Irvine, CA}

    \resumesection{Skills}
        \skillsentry{Languages}{Python, Java, MySQL, C++, C, LaTex}
        \skillsentry{Tools}{ROOT, RooFit, scikit-learn, git, SVT, Linux}

\end{document}
